\begin{verbatim}

\end{verbatim}
\begin{LARGE}\textbf{Zusammenfassung}\end{LARGE}\\
\begin{verbatim}

\end{verbatim}
Ziel der Bachelorarbeit ist es, eine Algorithmik zu entwickeln, die auf der Basis von Kamerabildern
Objekte am Meeresboden erkennt und deren Lage bestimmt. Neben den Bildern der Kamera sollen
auch a priori Informationen, wie Größe der Objekte oder deren grobe Lage verwendet werden.
Zusätzlich zur Detektion der Objekte soll eine Schätzung des Objektverlaufs, bei Nichtdetektion der
Objekte implementiert werden.
Zur Entwicklung wird ein AUV in einer bereits vorhandenen Simulationsumgebung verwendet.

\vfill

\begin{LARGE}\textbf{Abstract}\end{LARGE}\\
\begin{verbatim}

\end{verbatim}
In this bachelor thesis a camera-based and tracking system for oblong objects on the seafloor will be developed. Beside the raw images a-priori knowledge such as the diameter of the tracked object or its expected path should be used.\\
Additionally a system for estimating the object path should make a tracking possible even in case of missing detection results.
For the developing process a simulation environment will be used.
\vfill