%\addGlossaryEntry{glossary:Begriff}{
%    name={Begriff},
%    description={Beschreibung des Begriffs.}
%}
\addGlossaryEntry{vp}{
    name={Vanishing point},
    description={Punkt, an dem sich parallele Linien projziert auf die Bildebene schneiden}
}
\addGlossaryEntry{roll}{
    name={Roll},
    description={\textit{Rollen} des Fahrzeugs. Gibt eine Rotation um die $X-Achse$ (Längsachse) des Bodykoordinatensystems an.}
}
\addGlossaryEntry{pitch}{
    name={Pitch},
    description={\textit{Nicken} des Fahrzeugs. Gibt eine Rotation um die $Y-Achse$ (Querachse) des Bodykoordinatensystems an.}
}
\addGlossaryEntry{yaw}{
    name={Yaw},
    description={\textit{Heading} des Fahrzeugs. Gibt eine Rotation um die $Z-Achse$ (horizontale Achse) des Bodykoordinatensystems an.}
}
\addGlossaryEntry{lanefollow}{
    name={Lane follower controller},
    description={}
}
\addGlossaryEntry{wgs84}{
    name={WGS84},
    description={}
}
\addGlossaryEntry{pose}{
    name={Pose},
    description={Gibt die räumliche Lage eines Objektes an. Die Lage wird als Kombination aus Position und Orientierung angegeben.}
}
\addGlossaryEntry{lcf}{
    name={LCF},
    description={\textit{Lane-curve-function}. Ein parametrisierbares Modell, dass einen geraden Beginn mit einer anschließenden Krümmung darstellt.}
}
\addGlossaryEntry{rans}{
    name={RANSAC},
    description={\textit{Random sample consensus}. Iterativer Algorithmus, der durch zufälliges auswählen von Parametern ein geeignetes Modell für gegebene Daten schätzt.}
}
\addGlossaryEntry{transform}{
    name={Transformation},
    plural={Transformationen},
    description={Umrechnung von Koordinaten eines Punktes, in dieser Arbeit bestehend aus Translation (Verschiebung) und Rotation.}
}
\addGlossaryEntry{blur}{
    name={Bewegungsunschärfe},
    description={}
}