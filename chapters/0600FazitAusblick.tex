\section{Fazit \& Ausblick}
In diesem Kapitel wird ein Fazit zu den Ergebnissen der Arbeit gezogen. Hierbei wird auch auf die Einsatzfähigkeit der Lösung auf realen Testsystemen eingegangen.\\
Zuletzt wird noch ein Ausblick auf mögliche Folgearbeiten gegeben.
\subsection{Fazit}
Die in der Arbeit vorgestellte Lösung erzielt, wie die Tests zeigen, gute Ergebnisse. Gemäß der Zielsetzung ist es möglich, verschiedensten Objektverläufen zuverlässig zu folgen. Wie die Tests zur Objektdetektion zeigen, ist die Detektion auch unter schlechten Sichtbedingungen möglich. Auch die Verfolgung der Objekte funktioniert dabei zufriedenstellend. Einzig die Störanfälligkeit ist hierbei größer. Es hat sich gezeigt, dass die Fahrzeugregelung bei stetig wechselnden Wegpunkten nicht optimal ist, was in einigen Fällen zu einem schlechten Abfahren des eigentlich guten Polynoms führt. Trotzdem kann das verwendete Verfahren die nicht optimale Polynomverfolung durch gute Extrapolation in den meisten Fällen ausgleichen. Da der Fokus der Arbeit nicht auf der Fahrzeugregelung liegt und auch kein voller Zugriff darauf besteht, ist der Ausgleich durch die Extrapolation ein gewünschtes Verhalten.\\
Weniger optimal ist jedoch die Abhängigkeit des Verfahrens von der Parametrisierung. Der erste gut zu wählende Parameter ist der Templateschwellenwert für die Objekterkennung. Ist dieser zu gering, ist das Binärbild zu sehr von Störpunkten gefüllt, ist er zu hoch, wird das Objekt nicht mehr gut im Binärbild abgebildet. Beides führt zu einer schlechteren Detektion und somit zu einem schlechteren Gesamtverhalten.\\
Für die Parametrisierung des Schätzverfahrens wird viel Vorwissen über die Objekte benötigt. Nur wenn bekannt ist, ob das Objekt stark gekrümmt ist oder nicht, kann eine gute Verfolgung gesichert werden. Bei sich oft ändernden geometrischen Formen der Objekte kann nur sehr schwer eine geeignete Parametrisierung für alle Abschnitte gefunden werden.\\
Die naheliegendsten Anwendungen liegen in der Untersuchung von menschlich erzeugten Strukturen, wie Ölpipelines oder Unterwasserkabeln. Betrachtet man diese zwei Szenarien, ist das benötigte Vorwissen über die Strukturen durchaus gegeben. Bei Pipelines ist alleine durch den Durchmesser kein hoher Krümmungsgrad gegeben. Eine Parametrisierung ist also möglich.\\

Für einen Einsatz auf einem Livesystem ist die Laufzeit der Komponenten ebenfalls richtungsweisend. Mit den Angaben aus Kapitel \ref{sec_laufzeit} lässt sich eine ungefähre maximale Fahrtgeschwindigkeit abschätzen. Rechnet man mit einer Höhe von $6$m über Grund und einem Sichtwinkel von $45^\circ$ voraus kann ein Objekt in bis zu $6$m Entfernung erkannt werden. Da ein \gls{auv} mit zum Halten der Zielhöhe stets leicht nach vorne gebeugt fahren muss, sind die $6$m nicht ganz zu erreichen. Um eine Kurve rechtzeitig zu entdecken und schnell genug einen geeigneten Wegpunkt zu Berechnen sollten zwei Folgebilder sich überschneiden.\\
Bei einer Gesamtlaufzeit der Komponenten von ca. $2$ Sekunden kombiniert mit den vorangegangenen Überlegungen ergibt dies eine maximale Fahrgeschwindigkeit von ungefähr $1,5 \frac{\texttt{m}}{\texttt{sek}}$ bzw. ca. $3$ Knoten. In einem Szenario, in dem Pipelines oder Kabel inspiziert werden sollen ist auch keine höhere Geschwindigkeit erforderlich. Somit ist das implementierte Verfahren durchaus für einen Einsatz im Realsystem geeignet (siehe Kapitel \ref{sec_real}).

\subsection{Ausblick}

\subsubsection{Parametrisierung}
Da die Parametrisierung einen entscheidenden Einfluss auf das Verfahren hat, könnten folgende Arbeiten an dieser Stelle mehr Flexibilität bringen. Zum Beispiel kann ein Verfahren entwickelt werden, dass den Templateschwellenwert zur Binärisierung während der Objektdetektion dynamisch anpasst. In einer echten Mission können sich im Verlauf die Sichtbedingungen ändern. Ein fest gesetzter Schwellenwert wird in so einem Fall keine Ergebnisse mehr liefern können. Der Schwellenwert könnte bei einem über dem gesamten Bild niedrigen Rotwert gesenkt, bei vielen Störpunkten erhöht oder bei längerem Ausbleiben von Detektionsergebnissen wiederum gesenkt werden.\\

Ein ähnliches Prinzip kann auch für die Parameter des Schätzverfahrens angewendet werden. Ein Verfahren kann unabhängig von der Regression den aktuellen Verlauf im Bezug auf die Parametrisierung zu schätzen. Dieser Ansatz kann beispielsweise als Klassifizierungsproblem definiert werden und je nach Objektklasse ein definiertes Parameterset genutzt werden.

\subsubsection{Regressionsverfahren}
\label{sec_learnWeights}
In der implementierten Lösung werden die Daten für die Regression nur quantitativ gewichtet. Sollte es also viele Fehldetektionen in einem Bereich geben, in dem das Objekt unsichtbar ist, würden diese Fehler das Schätzverfahren stark beeinflussen und die Verfolgung gefährden.\\
Hier ist eine qualitative Bewertung der Daten eine gute Lösung. Es kann bewertet werden, wie \textit{sicher} die Objekterkennung eine Detektion als Teil des Objektes erkennt. Eine Grundlage für eine solche Bewertung wird von der implementierten Objekterkennung ausgegeben. In Listing \ref{pointInFrame} sind die Gütefaktoren angegeben. Die \texttt{peakheight} gibt an, wie groß der Unterschied der Rotwerte der Pixel des Objektes im Vergleich zum Durchschnitt des gesamten Bildes ist. \texttt{Area} bestimmt, wie viel der erwarteten Objektfläche das detektierte Objekt ausfüllt. \texttt{relativeCount} gibt das Verhältnis der Inlier nach dem \gls{rans} zur Gesamtanzahl der Punkte im Binärbild an. \texttt{fitsBorder} zeigt als Boolean, ob das detektierte Objekt zur erwarteten Breite passt. \texttt{numParts} gibt an in wie vielen Bildsegmenten das Objekt erkannt wurde. \texttt{theta} und \texttt{phi} geben die Neigungswinkel des \gls{auv}s zum Zeitpunkt der Detektion an.\\
Mit diesen Eigenschaften als Grundlage kann ein \textit{machine learning} Verfahren trainiert werden. Hierfür können Trainingsdaten bestehend aus den Eigenschaften und dem Fehler von jedem Punkt zum echten Objekt herangezogen werden. Nach dem Training kann dann aus der Objekterkennung ein Fehlerausschlag erwartet und ebenso eine qualitative Bewertung für die Regression durchgeführt werden. \todo{werden... werden ... werden}

\subsubsection{Integration in ein Realsystem}
\label{sec_real}
Ein weiterer Schritt ist die Integration der Lösung auf dem Realsystem. Hierfür kann die komplette Lösung inklusive Fahrzeugregelung aus Simulink exportiert werden. Dabei ist auch eine Generierung von C/C++-Code möglich, was zu einer Verbesserung der Laufzeit führen kann.\\
Bevor jedoch die Lösung exportiert werden kann, muss das Regressionsverfahren angepasst werden. Da die Regression auf der \textit{Optimization-Toolbox} basiert, welche aber nicht exportiert werden kann, muss hier eine gleichwertige Alternative entweder in \matlab oder aber auch in Form einer C oder C++ Bibliothek gefunden werden.
Ebenfalls kann in diesem Schritt die Regelung angepasst werden, sodass ein allgemein ruhigerer Fahrtverlauf ermöglicht werden kann. Gerade, wenn das \gls{auv} als Sensorträger verwendet wird, ist eine ruhige Fahrt fördernd für verlässliche Sensordaten während der Mission.