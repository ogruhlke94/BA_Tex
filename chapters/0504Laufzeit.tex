\subsection{Laufzeittests}
\label{sec_laufzeit}
Tests zur Laufzeit wurden mit dem \texttt{Simulink Profiler}\footnote{https://de.mathworks.com/help/simulink/ug/how-profiler-captures-performance-data.html} durchgeführt. Der \texttt{Simulink Profiler} erstellt nach einem Simulationsdurchlauf einen Report über Laufzeiten der einzelnen Komponenten. Die wichtigen Daten, die für Aussagen über Laufzeit aus diesem Report sind die Anzahl der Aufrufe und der \textit{self time} (Laufzeit) der Komponente.\\
Die Analyse der aufgezeichneten Daten während mehrerer Testläufe zeigte, dass die Objekterkennung und das Schätzverfahren die einzigen für die Laufzeitbetrachtung relevanten Komponenten sind. Andere Komponenten, wie die Berechnung der Wegpunkte oder die Transformationen fallen hingegen nicht ins Gewicht und benötigen bei einer Gesamtlaufzeit der Simulation von $1000$ Sekunden lediglich ca. $2$ Sekunden. 

\begin{verbatim}

\end{verbatim}
Die Daten stammen aus Testläufen zwischen $1000$ und $2500$ Sekunden Gesamtsimulationszeit. Dies entspricht zwischen $180$ und $350$ Komponentenaufrufen.\\
Die Objekterkennung benötigte hierbei im Durchschnitt ($\frac{Komponentenlaufzeit}{Anzahl Aufrufe}$) zwischen $1,3$ und $1,56$ Sekunden.\\
Das Schätzverfahren benötigte im Schnitt zwischen $0,65$ und $0,8$ Sekunden.\\

Zu diesen Angaben muss noch beachtet werden, dass Simulink an einigen Stellen die Arbeit limitierte und auf verlangsamende Alternativen zurückgegriffen werden musste. Zum Beispiel ist das Schätzverfahren in einem \textit{Interpreted MATLAB Function Block}\footnote{https://de.mathworks.com/help/simulink/slref/interpretedmatlabfunction.html} umgesetzt. Bereits in der Dokumentation des Blocks steht die Anmerkung: \textit{"This block is slower than the Fcn block because it calls the MATLAB parser during each integration step. Consider using built-in blocks (such as the Fcn block or the Math Function block) instead"}. Dieser Block ist jedoch die einzige einfache Möglichkeit anonyme Funktionen zu benutzen, die für die Optimierungsfunktionen der \textit{Optimization Toolbox} zwingend notwendig sind, um selbst entworfene Probleme zu lösen.

%ip
%179 240.18 = 1.34
%196 310.24 = 1.58
%351 548.60 = 1.56
%163 224.84 = 1.38
%159	241.81 = 1.52
%
%cf
%179 122.04 = 0.68
%196 132.63 = 0.67
%351 273.55 = 0.78
%163 106.28 = 0.65
%159 103.22 = 0.65