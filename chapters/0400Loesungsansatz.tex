\cleardoublepage
\section{Lösungsansatz}
In diesem Kapitel wird der von mir implementierte Lösungsansatz vorgestellt. Grundlegend gehe ich zuerst auf die Erweiterung der Simulationsumgebung und die implementierte Transformationskette ein.\\
Danach folgen genauere Ausführungen zu den Kernelementen der Arbeit, der Objekterkennung und dem Schätzverfahren.\\

Ein grundlegender Datentyp, der sich über alle Teile der Arbeit erstreckt, ist durch die Struktur \textit{pointInFrame} [Listing \ref{pointInFrame}] definiert. Diese Struktur bildet einen von der Objekterkennung detektierten Punkt des Objektes mit seiner Position und Orientierung ab. Zudem wird gespeichert, in welchem Referenzframe der Punkt angegeben ist. Neben diesen Positionsdaten sind zudem noch die Gütefaktoren der Objekterkennung angegeben (siehe hierfür Kapitel \ref{sec_learnWeights}).

\begin{lstlisting}[language=Matlab,caption={[Initialisierung der \textit{pointInFrame}-Struktur]Initialisierung der \textit{pointInFrame}-Struktur, die erkannte Punkte in verschiedenen Referenzkoordinatensystemen abbildet.}, label = pointInFrame]
Point_In_Frame = struct;
Point_In_Frame.point = [0 0 0];
Point_In_Frame.direction = 0;
Point_In_Frame.peakheight = 0;
Point_In_Frame.area = 0;
Point_In_Frame.frame = frames.image
Point_In_Frame.numParts = 0;
Point_In_Frame.fitsBorder = false;
Point_In_Frame.relativeCount = 0;
Point_In_Frame.valid = false;
Point_In_Frame.theta = 0;
Point_In_Frame.phi = 0;
\end{lstlisting}

\begin{figure}[H]
\includegraphics[width=\textwidth]{Systementwurf.png}
\caption[Systementwurf]{Systementwurf der eingesetzten Lösung. Die Simulationsumgebung ist hierbei als \textit{black box} zu betrachten, die das Kamerabild und \gls{auv}-Pose liefert und durch die Wegpunkte für den \gls{lanefollow} gesteuert wird. In dieser Grafik ist der Datenfluss der vier Hauptkomponenten der Arbeit (Objekterkennung, Transformation, Schätzverfahren und Wegpunktberechnung) dargestellt.}
\end{figure}
Die Objekterkennung nutzt das Kamerabild aus der Simulationsumgebung um in zwei Schritten (Binärisierung und \gls{rans}) das Objekt zu detektieren. Die detektierten Punkte werden mithilfe der Information der \gls{auv}-Pose in das globale Weltkoordinatensystem transformiert. Im Schätzverfahren wird das alternative Weltkoordinatensystem verwendet, in welchem auch die Regression durchgeführt wird. Mit dem Polynom aus dem Schätzverfahren und der \gls{auv}-Pose werden die Wegpunkte für den \gls{lanefollow} bestimmt.

\subsection{Simulationserweiterung}
\subsubsection{Steuerung}
\label{sec_waypoint}
Wie im Kapitel \ref{sec_auvSimGrundlage} Grundlagen beschrieben, wird in der bestehenden Simulation ein \texttt{\gls{lanefollow}}, der eine Linie zwischen einem \textit{old\_waypoint} und einem \textit{new\_waypoint} bildet, verwendet.
Die Schnittstelle zur Steuerung bildet somit die Kombination aus den beiden Wegpunkten. Die Berechnung der Wegpunkte wird auf Basis des Polynoms aus dem Schätzverfahren generiert.\\
Zunächst wird die Position des \gls{auv}s durch die aktuelle Transformationsmatrix in das alternative Weltkoordinatensystem (siehe Absatz \ref{alterWorldCoords}) transformiert. Es wird der, zu dieser Position, nächstgelegene Punkt auf dem Polynom berechnet. Dieser Punkt dient als Zentrum für einen Kreis zur Bestimmung der Wegpunkte. Mithilfe der Kreisgleichung [Gleichung \ref{circEq}] werden die zwei Schnittpunkte des Polynoms mit dem Kreis berechnet. Da durch die Nutzung des alternativen Weltkoordinatensystem sichergestellt wird, dass das \gls{auv} in Richtung der X-Achse fährt, kann problemlos der Schnittpunkt mit höherem $x$-Wert als \textit{next\_waypoint} und dementsprechend der zweite als \textit{old\_waypoint} verwendet werden. Es wird davon ausgegangen, dass bei einem solch kleinen Kreisradius (zwischen 5 und 10 Metern) nicht mehr als zwei Schnittpunkte zwischen Polynom und Kreis vorhanden sind. Sollte dies der Fall sein, wäre das Polynom viel zu stark gekrümmt, um noch verfolgt zu werden. Im Szenario dieser Arbeit gibt es auch keine Objekte, die eine solch starke Krümmung aufweisen.\\
Der letzte Schritt besteht aus der \gls{transform} der Wegpunkte in das reale VRML-Koordinatensystem mithilfe der inversen Transformationsmatrix.
Das Verfahren ist in Abbildung \ref{wpCircle} grafisch dargestellt.

\begin{ownequation}[H]
\begin{equation}
0 = (X_{test}-Center_X)^2+(Y_{test}-Center_Y)^2 - r^2
\end{equation}
\caption[Kreisgleichung zum Test, ob ein Punkt auf einem Kreis liegt.]{Kreisgleichung zum Test ob ein Punkt $X_{test},Y_{test}$ auf einem Kreis liegt. $Center_X$ und $Center_Y$ bilden hierbei den Mittelpunkt eines Kreises mit Durchmesser $r$.}
\label{circEq}
\end{ownequation}

\begin{figure}[H]
\centering
\includegraphics[scale=0.7]{waypointProvier.jpg}
\caption[Verfahren zur Bestimmung der Wegpunkte]{Verfahren zur Bestimmung der Wegpunkte. Der Wegpunkt wird im alternativen Weltkoordinatensystem bestimmt. Hierbei wird ein Kreis um den nächsten Punkt vom \gls{auv} auf dem Polygon bestimmt. Die Schnittpunkte des Kreises bilden die Wegpunkte für die \gls{auv}-Steuerung.}
\label{wpCircle}
\end{figure}
\subsubsection{Kamerabilder}
Da die Simulation in der ursprünglichen Form noch sehr \textit{klinische} Bilder generierte, mussten diese Bilder künstlich verschlechtert und die Sichtverhältnisse eingeschränkt werden, um realistische Eingangsbilder zu erzeugen. In Abbildung \ref{simPics} ist von links nach rechts ein ursprüngliches Kamerabild, ein verschlechtertes Bild und ein sehr stark verschlechtertes Bild zu sehen. Die Testläufe der Arbeit wurden mit dem Verschlechterungsgrad des mittleren Bildes durchgeführt. Die Objekterkennung wurde zudem noch mit Bildern, wie dem rechten Bild getestet.
\begin{figure}[H]
\centering
\begin{tabular}{cc}
\subfloat[Ursprüngliches Bild]{\includegraphics[height=0.25\textheight,width=0.4\textwidth]{/imageProcessing/gradeOptimal.jpg}}&
\subfloat[Bild verschlechtert mit leichter \gls{blur} und geringem Pixelrauschen]{\includegraphics[height=0.25\textheight,width=0.4\textwidth]{/imageProcessing/graeOk.jpg}}\\
\subfloat[Sichtverhältnisse stark verschlechtert mit \gls{blur} und geringem Pixelrauschen]{\includegraphics[height=0.25\textheight,width=0.4\textwidth]{/imageProcessing/gradeTestQuali.jpg}}&
\subfloat[Sichtverhältnisse sehr stark verschlechtert, simulierte Reflexion des Wassers mit \gls{blur} und geringem Pixelrauschen]{\includegraphics[height=0.25\textheight,width=0.4\textwidth]{/imageProcessing/gradeschlecht.jpg}}\label{img_badSeight}
\end{tabular}
\caption[Simulationsbilder]{Simulationsbilder. \textit{a)} zeigt das ursprüngliche Bild. In \textit{b)} bis \textit{e)} wird das Bild auf verschiedenen Arten verschlechtert. Die meisten Testläufe wurden mit der Verschlechterung von \textit{b)} und \textit{d)} durchgeführt. Tests unter sehr schlechten Bedingungen mit einem Grad der Verschlechterung aus \textit{e)}}
\label{simPics}
\end{figure}
\newpage
\subsection{Transformation}
\label{sec_transformations}
Wie bereits in der Einleitung beschrieben, werden mehrere Koordinatensysteme genutzt. Zur sicheren Verwendung der Koordinatensysteme sind \glspl{transform} unter den Systemen zwingend nötig.
Eine \gls{transform} besteht aus einer Rotation und einer Translation, die sich aus den Beziehungen der Systeme zueinander ergibt.\\
Im \texttt{enum} \textit{frames} [Listing \ref{framesEnum}] sind die verschiedenen Frames definiert, zwischen denen eine \gls{transform} möglich ist.

\lstinputlisting[language=Matlab,caption={Enumeration der Frames,die die verschiedenen Koordinatensystem bezeichnen},label=framesEnum]{frames.m}

Umgesetzt wird eine \gls{transform} aus einem \textit{source} Frame in einen \textit{target} Frame durch die Funktion \texttt{transform} [Listing \ref{transform}]. Die Transformation ist nur in eine Richtung möglich, da die inverse Transformation für diese Arbeit nicht benötigt wurde.

\lstinputlisting[language=Matlab,caption={\gls{transform} von \textit{source} in \textit{target} Frame. Die verschiedenen Transformation werden hier verwaltet und solange die nächste Transformation ausgeführt, bis der \textit{target} Frame erreicht wird.},label=transform]{transform.m}

\subsubsection{Bild zu Kamera}
\label{section_PicToCam}
Die verlustfreie \gls{transform} von 2D-Pixelkoordinaten in 3D-Kamerakoordinaten ist mit einer Kamera nicht möglich, da die Tiefeninformation nicht vorhanden ist. Jedoch lässt sich mit dem Wissen über die Entfernung zur Bildebene, der \textit{flat world assumption} (siehe Kapitel \ref{sec_coordsystems}) und den intrinsischen Kameraparametern eine ausreichend gute Transformation durchführen. Da die Kamera gerade nach unten gerichtet ist, entspricht die Entfernung zur Bildebene der Höhe des \gls{auv}s über dem Meeresboden, welche über die Sensorik bestimmt wird. Die intrinsischen Kameraparameter lassen sich über eine Kamerakalibrierung bestimmen, welche mithilfe der \matlab \textit{Computer Vision System Toolbox} durchgeführt.\\
Da die resultierende \gls{transform} am besten im Abstand der Kalibrierung funktioniert wurde die Kalibrierung in einem Abstand von 6 Metern durchgeführt, was im späteren Verlauf auch der gewünschte Abstand zum Boden ist.\\
Aus der Kamerakalibrierung wird ein \textit{CameraParameter}\footnote{https://de.mathworks.com/help/vision/ref/cameraparameters-class.html} Objekt erzeugt, welches die Methode \textit{pointsToWorld} bietet. Die Methode berechnet eine Projektionsmatrix aus den Kamera-Parametern und dem bekannten Abstand der Kamera zum Objekt. Mithilfe der Inversen dieser Matrix können dann Pixel in Kamerakoordinaten umgerechnet werden.\\
Leichte Neigungswinkel, die während der Fahrt auftreten, können durch die Multiplikation mit der entsprechenden Rotationsmatrix herausgerechnet werden. Jedoch ist dabei zu beachten, dass durch die Neigungswinkel die Fläche, die die Kamera sieht, vergrößert wird. Dadurch bilden einzelne Pixel mehr Fläche ab und die \gls{transform} wird ungenauer.\\
Die $z$-Koordinate ergibt sich aus dem Wissen, Objekte am Meeresboden zu betrachten und der Tatsache, dass die Höhe der Kamera über dem Meeresboden bekannt ist.

\subsubsection{Kamera zu Body}
Die \gls{transform} vom Kamerakoordinatensystem zum Bodykoordinatensystem besteht aus einer Translation und einer Rotation, die durch die Montageposition der Kamera am \gls{auv} bestimmt wird [Kapitel \ref{sec_img_cam_coords}].\\
Aufgrund der Verschiebung der Kamera zum Bodykoordinatenursprung (Schwerpunkt des \gls{auv}s) ergibt sich eine Translation um $1,3$m in x-Richtung und $0,25$m in z-Richtung.\\
Die Rotation beträgt dabei $90^\circ$ um die Z-Achse.\\
Somit ergibt sich die Tranformationsmatrix aus Gleichung \ref{trans_cam_body}\\

\begin{ownequation}[H]
\begin{equation}
\begin{pmatrix}
x_{body}\\y_{body}\\z_{body}\\1
\end{pmatrix}
=
\begin{pmatrix}
0 & -1 & 0& 1,3\\
1 & 0 & 0& 0\\
0 & 0 & 1& 0,25\\
0 & 0 & 0 & 1
\end{pmatrix}
\cdot
\begin{pmatrix}
x_{cam}\\y_{cam}\\z_{cam}\\1
\end{pmatrix}
\end{equation}
\caption[\gls{transform} der Kamerakoordinaten zu Bodykoordinaten]{Transformation der Kamerakoordinaten zu Bodykoordinaten. Die Kamerakoordinaten werden um $1,3$m auf der x-Achse und $0,25$m auf der z-Achse verschoben. Außerdem wird eine Rotation um $90^\circ$ um die Z-Achse durchgeführt.}
\label{trans_cam_body}
\end{ownequation}

\subsubsection{Body zu Welt}
Für die \gls{transform} vom Bodykoordinatensystem in das Weltkoordinatensystem ist wieder eine Translation und eine Rotation nötig.
Aus der Definition der Koordinatensysteme folgt zunächst eine Rotation um $180^\circ$ um die X-Achse nötig.
Die Translation ergibt sich aus der Position des \gls{auv}s (Position Nord/Ost in Metern).\\
Die Rotation wird durch die Ausrichtung des \gls{auv}s in der Welt (\gls{yaw} [Abb. \ref{Abb. 2}]) bestimmt. Somit ergibt sich die Tranformationsmatrix aus Gleichung \ref{trans_body_world}\\

\begin{ownequation}[H]
\begin{equation}
\begin{split}
\begin{pmatrix}
x_{world} \\ y_{world} \\ z_{world} \\ 1
\end{pmatrix}
& =
\begin{pmatrix}
cos(yaw) & -sin(yaw) & 0 & Pos_{north}\\
sin(yaw) & cos(yaw) & 0 & Pos_{east}\\
0 & 0 & 1 & 0\\
0 & 0 & 0 & 1
\end{pmatrix}\\
&\cdot
\left(
\begin{pmatrix}
1 & 0 & 0& 0\\
0 & -1 & 0& 0\\
0 & 0 & -1& 0\\
0 & 0 & 0 & 1
\end{pmatrix}
\cdot
\begin{pmatrix}
x_{body} \\ y_{body} \\ z_{body} \\ 1
\end{pmatrix}
\right)\\
\end{split}
\end{equation}
\caption[\gls{transform} der Bodykoordinaten zu Weltkoordinaten]{Transformation der Bodykoordinaten zu Weltkoordinaten. Zunächst werden die Body-Koordinaten um $180^\circ$ um die X-Achse rotiert. Im Anschluss findet eine Translation zu der Position des \gls{auv}s in der Welt und eine Rotation um die Z-Achse, die die Ausrichtung des \gls{auv}s abbildet, statt.}
\label{trans_body_world}
\end{ownequation}

\subsubsection{Welt zu \vrml}
Für die \gls{transform} von Weltkoordinaten in \vrml Koordinaten ist nur eine Rotation um $-90^\circ$ um die X-Achse nötig [Abb. \ref{trans_world_vrml}].
\begin{ownequation}[H]
\begin{equation}
\begin{pmatrix}
x_{vrml}\\y_{vrml}\\z_{vrml}\\1
\end{pmatrix}
=
\begin{pmatrix}
1 & 0 & 0& 0\\
0 & -1 & 0& 0\\
0 & 0 & -1& 0\\
0 & 0 & 0 & 1
\end{pmatrix}
\cdot
\begin{pmatrix}
x_{body}\\y_{body}\\z_{body}\\1
\end{pmatrix}
\end{equation}
\caption[\gls{transform} von Weltkoordinaten in \vrml Koordinaten]{Transformation von Weltkoordinaten in \vrml Koordinaten. Hierfür ist nur eine Rotation um $-90^\circ$ um die X-Achse nötig.}
\label{trans_world_vrml}
\end{ownequation}
\newpage
\subsection{Objekterkennung}
\label{sec_objDet}
In diesem Kapitel wird die umgesetzte Objekterkennung beschrieben. Dabei wird aus einem Rohbild der Kamera ein \textit{pointInFrame}-Objekt erzeugt. Grob besteht die Objekterkennung aus zwei Schritten. Zuerst wird aus dem Rohbild ein Binärbild erzeugt und im Anschluss im Binärbild das gesuchte Objekt detektiert.
\subsubsection{Binärbild mit Template}
\label{sec_templ}
\todo{Kann man die graphen mit legends gut genug erkennen?}
Die Objekterkennung basiert auf einem ähnlichen Verfahren wie das vorgestellte CSurvey Projekt\cite{Albiez2015CSurveyA}.\\
Da eine Farberkennung aufgrund der Sichtbedingungen nicht in Frage kommt, wird im ersten Schritt das RGB-Bild in ein Graustufenbild umgewandelt. Der erste Ansatz bestand darin, das Helligkeitsbild zu betrachten, da ein gesuchtes Objekt einen höheren Helligkeitswert besitzt als der Meeresboden (siehe Abb. \ref{brightCurve_real} und \ref{brightCurve_sim}).\\
Aus Erfahrungswerten früherer Projekte riet Christopher Gaudig mir, die Rotwerte der Bilder zu betrachten, da oftmals sowohl der Meeresboden als auch trübes Wasser geringe Rotwerte haben. In den Abbildungen \ref{redCurve_real} und \ref{redCurve_sim} ist dies zu beobachten. Die Kurven sehen denen der Helligkeitswerte sehr ähnlich, jedoch sind die Ausschläge des Objektes in den Rotwerten höher.\\
Im nächsten Schritt wird mithilfe eines Templates [Abb. \ref{templImg}] ein Binärbild erzeugt. Das Template zeichnet sich durch drei Pixelangaben aus. Die \textit{Testpixel} (rot) geben einen Bereich an, der im aktuellen Schritt geprüft wird. Die \textit{Checkpixel} (blau) geben den Bereich rechts und links neben dem Testbereich an und bilden den Referenzwert. Die \textit{Borderpixel} (grün) geben einen Bereich zwischen Test- und Checkbereich an, der ignoriert wird. Durch das Ignorieren des Bereichs kann ein langsamer regelmäßiger Übergang, der bei Betrachtung der direkten Nachbarpixeln des Testbereichs als Checkbereich zu einem geringen Templatewert führt, trotzdem noch einen hohen Templatewert ergeben. Jedes Pixel dient einmal als Mittelpunkt des Testbereichs, um zu entscheiden, ob das betrachtete Pixel Teil des Objektes sein kann. Dies ist der Fall, wenn der Wert des Pixels [Gleichung \ref{templateValue}] einen Schwellenwert (rote Linie) übersteigt.\\
%\begin{ownequation}[H]
%\begin{eqnarray}
%TP_i = \left\{i-\frac{\#TP}{2} \dots i \dots i+\frac{\#TP}{2}\right\}\\
%LC_i = \left\{i-\frac{\#TP}{2}-\#BP-\#CP \dots i-\frac{\#TP}{2}-\#BP\right\}\\
%RC_i = \left\{i-i+\frac{\#TP}{2}+\#BP \dots \frac{\#TP}{2}+\#BP+\#CP\right\}\\
%\end{eqnarray}
%\caption{Testbereich ($TP$) und Checkbereiche ($LC$ und $RC$) für ein Pixel mit Index $i$}
%\end{ownequation}
%Mit diesen Mengen lässt sich dann die Berechnung eines Templatewerts ($TV$) zu definieren.
%\begin{ownequation}[H]
%\begin{eqnarray}
%TV_i = \frac{\sum_{x \in TP_i} image(x))}{\#TP} - \left(\frac{\sum_{y_1 \in LC_i} image(y_1) + \sum_{y_2 \in RC_i} image(y_2)}{2 \cdot \#CP}\right)
%\end{eqnarray}
%\caption{Templatewertberechnung für ein Pixel $i$}
%\label{templateValue}
%\end{ownequation}

\begin{figure}[H]
\centering
\includegraphics[scale=0.5]{imageProcessing/Prinzip/template.jpg}
\caption[Template zum Bestimmen des Binärbilds]{Template zum Bestimmen des Binärbilds. Getestet wird das Pixel im Zentrum des Testbereichs (rot). Der Templatewert ergibt sich aus der Subtraktion des Durchschnitts im Checkbereich (blau) vom Durchschnitt des Testbereichs (rot). Der Borderbereich (grün) wird dabei nicht beachtet.}
\label{templImg}
\end{figure}

\begin{ownequation}[H]
\begin{equation}
TV = \frac{sum(Testpixel)}{\#TP} - \frac{sum(Checkpixel)}{\#CP}
\end{equation}
\caption[Templatewertberechnung für ein Pixel als Formelausdruck]{Templatewertberechnung für ein Pixel als Formelausdruck.Der berechnete Wert $TV$ ist der Unterschied zwischen Test- und Checkbereich.$\#TP$ bezeichnet die Anzahl der Testpixel und $\#CP$ die Anzahl der Checkpixel.}
\label{templateValue}
\end{ownequation}

\begin{figure}[H]
\begin{tabular}{cc}
\multicolumn{2}{c}{\subfloat[Originalbild. Das Testbild stammt aus Aufnahmen eines Testlaufs im Unisee (siehe Abschnitt \ref{realObjTests}.]{\includegraphics[height=0.33\textheight,width=\textwidth]{imageProcessing/realPipe/003orgImstart.jpg}}}\\
\subfloat[Auswertung des Helligkeitsverlauf einer Bildzeile im oberen Drittel des Bildes]{\includegraphics[height=0.33\textheight,width=0.5\textwidth]{imageProcessing/Prinzip/hellReal.jpg}\label{brightCurve_real}}&
\subfloat[Auswertung des Rotwertverlauf einer Bildzeile im oberen Drittel des Bildes]{\includegraphics[height=0.33\textheight,width=0.5\textwidth]{imageProcessing/Prinzip/rotReal.jpg}\label{redCurve_real}}
\end{tabular}
\caption[Helligkeit und Rotwert im echten Testbild]{Helligkeit und Rotwert im echten Testbild. In beiden Grafiken zeigt die blaue Linie die jeweiligen Pixeldaten, die gelbe Linie den Wert des Testbereichs, die lila Linie den Durchschnitt des Checkbereichs und die weiter unten gelegene rote Linie den Templatewert für das Pixel. In beiden Templatewerten ist die Pipeline eindeutig zu erkennen, wobei der Ausschlag im Rotwert weitaus höher ist.}
\end{figure}

\begin{figure}[H]
\begin{tabular}{cc}
\multicolumn{2}{c}{\subfloat[Originalbild der Simulation]{\includegraphics[height=0.33\textheight,width=0.4\textwidth]{imageProcessing/gradeTestQuali.jpg}}}\\
\subfloat[Auswertung des Helligkeitsverlaufs einer Bildzeile im oberen Drittel des Bildes]{\includegraphics[height=0.33\textheight,width=0.5\textwidth]{imageProcessing/Prinzip/hellSim.jpg}\label{brightCurve_sim}}&
\subfloat[Auswertung des Rotwertverlaufs einer Bildzeile im oberen Drittel des Bildes]{\includegraphics[height=0.33\textheight,width=0.5\textwidth]{imageProcessing/Prinzip/rotSim.jpg}\label{redCurve_sim}}
\end{tabular}
\caption[Helligkeit und Rotwert im Simulationsbild]{Helligkeit und Rotwert im Simulationsbild.In beiden Grafiken zeigt die blaue Linie die jeweiligen Pixeldaten, die gelbe Linie den Wert des Testbereichs, die lila Linie den Durchschnitt des Checkbereichs und die weiter unten gelegene rote Linie den Templatewert für das Pixel. In beiden Templatewerten ist die Pipeline eindeutig zu erkennen, wobei der Ausschlag im Rotwert weitaus höher ist.}
\end{figure}
\todo{neu machen}

\subsubsection{\rans auf Binärbild}
Im weiteren Verlauf wird auf dem  Binärbild gearbeitet. Nach dem Vorbild von Wang et al. \cite{wang2004lane} wird das Bild in drei Segmente unterteilt.\\
In jedem Segment wird dann mithilfe des \rans -Algorithmus ein Rechteck gesucht [Listing \ref{ransPseudo}]. Der Algorithmus sampled verschiedene Rechtecke im Segment. Jedes Rechteck wird durch einen Mittelpunkt, eine Orientierung, die Breite und die Höhe definiert. Die Höhe ergibt sich aus der Höhe des Segmentes und die Breite wird durch die erwartete Breite des Objektes festgelegt. Mittelpunkt und Orientierung werden in jedem Iterationsschritt zufällig gewählt.\\
Für jeden Punkt des Binärbilds wird dann geprüft, ob er im Rechteck liegt (ein \textit{Inlier} ist). Gemäß des \rans wird das Rechteck mit den meisten Inliern gewählt.\\
\begin{lstlisting}[language=Matlab,caption={Eingesetzter \rans als Pseudocode.},label=ransPseudo]
function ransac(segment,height,width,minInlier,iterNum)
	maxInlier = 0;
	orientations = -pi/4:0.05:pi/4;
	bestCenter = None;
	bestOrientation = None;
	for i = 1:iterNum
		boxCenter = selectRandomPoint(segment);
		boxOrientation = selectRandomValue(orientations);
		inliers = findPointsInBox(segment, box=[boxCenter,boxOrientation,height,width]);
		
		if(len(inliers) > minInlier && len(inliers) > maxInlier)
			maxInlier = len(inliers)
			bestCenter = boxCenter;
			bestOrientation = boxOrientation;
		end
	end
end
\end{lstlisting}

Somit gibt es für jedes Bild bis zu drei Objektposen. Durch das Unterteilen in Segmente lässt sich zum einen bestimmen, in wie vielen Segmenten ein Objekt erkannt wurde (entspricht der \textit{Länge} des Objektes im Bild). Des weiteren kann ein gebogener Verlauf oder ein abgeknicktes Objekt im Bild sinnvoll erkannt werden.\\
In den ersten Tests dieses Verfahrens ist ersichtlich geworden, dass es einen \todo{richtige wortwahl?}Tradeoff zwischen Geschwindigkeit und Erkennungsgüte gab. Die Erkennung wurde besser, je größer das Maximum der Iteration für \rans gewählt wurden. Da der \rans jedoch auf jedes der drei Segmente separat angewendet wird, reduziert sich die Geschwindigkeit bei steigender Iterationsanzahl deutlich. Für eine zuverlässige Erkennung waren zu viele Iterationen nötig, sodass das Verfahren nicht einsetzbar wäre.\\
Als Lösung für dieses Probleme werden die möglichen erzeugten Rechtecke für den \rans begrenzt. Da das Template nur in horizontaler Richtung auf das Bild angewendet wird, sind horizontal liegende Objekte im Binärbild nicht sichtbar. Aufgrund dieser Tatsache lassen sich die Orientierungen auf einen Bereich begrenzen, anstatt diese komplett zufällig zu wählen. Durch diese Maßnahme wurden die benötigten Iterationen für ein zuverlässiges Ergebnis drastisch reduziert. Jedoch steigt auch die Gefahr Orientierungen nicht mehr richtig zu erkennen, wie in Abbildung \ref{detecFail} gezeigt.\\
\begin{figure}[H]
\centering
\includegraphics[scale=0.5]{imageProcessing/realPipe/004detectedImage.jpg}
\caption{Falsch detektierte Objektorientierung aufgrund der Beschränkung der Ausrichtungen für den \rans}
\label{detecFail}
\end{figure}
\todo{evtl box vom ransac zeigen}
Der zweite Faktor, der die Geschwindigkeit der Objekterkennung verringerte, ist die Menge der Punkte im Binärbild. So musste für jede Iteration des \rans für jeden Punkt geprüft werden, ob der Punkt im Rechteck liegt. In den Testbildern der Simulation lag die Anzahl der Punkte teilweise bei weit über $10000$, was in Kombination mit $200$ Iterationen zu einer inakzeptablen Laufzeit von ca. 5 Sekunden pro Bild führte.\\
Zum Lösen dieses Problems wurde vor dem Einsatz des \rans die Punktanzahl verringert, indem nur jedes dritte Pixel betrachtet wird und dieses den Mittelwert aller seiner Nachbarpixel erhält. \todo{Grafik?} Somit konnte die Punktanzahl zuverlässig auf unter $2000$ verringert werden, was zu einer deutlichen Beschleunigung (ca. 1 Sekunde pro Bild) ohne nennenswerte Verschlechterung der Ergebnisse führte.
%\subsubsection*{Gewicht der erkannten Punkte}
%\label{sec_weights}
%\begin{itemize}
%\item \textbf{numParts:} Länge des Objektes im Bild. Die erkannten Punkte werden nach Parallelität ihrer Ausrichtung und Nähe zueinander untersucht. Somit wird untersucht in wievielen Segmenten das Objekt fortgeführt wird.
%\item \textbf{area:} Um den Punkt wird ein Rechteck gelegt, dass der Breite des Objektes und der Höhe des Segmentes entspricht. \textbf{area} gibt einen relativen Wert an, wie ausgefüllt dieses Rechteck ist.
%\item \textbf{peakheight:} Es wird der Mittelwert aller der Helligkeit innerhalb des Rechtecks berechnet und mit dem allgemeinen Helligkeitswert des Bildes verglichen. \textbf{peakheight} dieses Verhältnis relativ an.
%\item \textbf{fitsBorder:} Gibt als boolean an, ob das erkannte Objekt der berechneten Objektbreite passt.
%\item \textbf{relativeCount:} Gibt das Verhältnis von Punkten im Binärbild, die zum Objekt passen und der Gesamtzahl der Punkte an.
%\end{itemize}
\newpage
\subsection{Schätzverfahren}
\label{sec_curveFit}
In diesem Abschnitt wird das implementierte Schätzverfahren erläutert. Das Verfahren nutzt die Erkenntnisse der Bilderkennung und versucht mithilfe der Regression ein Polynom $f$ zweiten \todo{Welcher Grad} Grades durch alle erkannten Punkte in Betrachtung ihrer Orientierung zu fitten.
Das Verfahren basiert auf dem \textit{Least-Squares} Verfahren\cite{simon2006optimal}, wobei versucht wird die Gleichung [\ref{lsq}] zu minimieren.\\
$x_i$ und $y_i$ sind hierbei die Koordinaten der erkannten Punkte. Es wird über alle Punkte summiert der quadratische Fehler vom Funktionswert zu gegebenen Parametern zum $y$ aus der Bilderkennung berechnet. $f(p,x)$ ist eine beliebige Funktion, die $x$ in Abhängigkeit von $p$ auf eine reelle Zahl abbildet.\\
\begin{ownequation}[H]
\begin{equation}
err = \sum_{i}(f(p,x_i)-y_i)^2
\end{equation}
\caption{Least Squares Verfahren}
\label{lsq}
\end{ownequation}
Die Ergebnisse der Bildverarbeitung können nicht alle gleich behandelt werden. Faktoren wie Sichtbedingung, Helligkeit des Bildes oder auch die Sichtbarkeit/Bedeckung des Objektes beeinflussen die Güte eines erkannten Punktes. Ebenso beeinflussen die Neigungswinkel des AUVs die Güte der Transformation (\ref{section_PicToCam}), somit sind auch Punkte die bei hohen Neigungswinkeln detektiert wurden schlechter zu bewerten.\\
Neben den qualitativen Merkmalen eines Punktes ist es ebenso sinnvoll alte Punkte über die Zeit schlechter zu Bewerten. Für eine gute Verfolgung eines Objektes sollten die zuletzt erkannten Punkte weitaus wichtiger sein, als Punkte die bereits mehrere Meter entfernt liegen.
Um diese Anforderungen umzusetzen habe ich eine Erweiterung des \textit{Least-Squares} genutzt, den \textit{Weighted-Least-Squares}[\ref{wlsq}]\\
\begin{ownequation}[H]
\begin{equation}
err = \sum_{i}w_i \cdot (f(p,x_i)-y_i)^2
\end{equation}
\caption{Weighted Least Squares Verfahren}
\label{wlsq}
\end{ownequation}

Das \textit{Weighted-Least-Squares} Verfahren bietet eine gute Grundlage für die Regression. Es bleiben jedoch noch einige Probleme, die das Verfahren in der Form nicht lösen kann.
\begin{enumerate}
\item Beachtung der Orientierung erkannter Punkte
\item Bedingungen für die Kurve (z.B. maximale Steigung)
\item Schätzungen, für Punktverläufe, die sich nicht durch eine einzelne Funktion darstellen lassen
\end{enumerate}

Zum Lösen der ersten zwei Probleme bietet die \matlab Funktion \textit{fmincon} eine geeignete Lösung. Die Funktion bietet die die Möglichkeit eine Funktion $F(p)$ zu minimieren, wobei mit $c(p) \leq 0$ eine Bedingung erfüllt werden muss. Die Funktion $c(p,x_i)$ [\ref{constraint}] berechnet über den Funktionsverlauf von $f(p,x)$ mithilfe der Ableitung $f'(p,x)$ die Steigung in jedem Punkt $x_i$. Da \textit{fmincon} prüft, ob die Bedingungsfunktion kleiner 0 ist wird von der Steigung ein Maximalwert ($max_{slope}$) abgezogen (\textit{Erfüllt 2.}).\\
\begin{ownequation}[H]
\begin{equation}
c(p,x_i) = f'(p,x_i)-max_{slope}
\end{equation}
\caption{Funktion zum überprüfen, ob die Steigung einen Maximalwert nicht übersteigt.}
\label{constraint}
\end{ownequation}
Die Funktion $F(p)$ wird als $F(p,x,y,s,w,n,m,tau)$ [\ref{minimizeFunction}] definiert, wobei $x$ und $y$ wieder die Punkte der Bilderkennung darstellen, $s$ die erkannte Orientierung im Punkt und $w$ das Gewicht der Punkte. Die Funktion $F$ besteht aus einer Linearkombination der Funktionen $g$ und $h$, wobei $g$ den summierten Fehler der Position [\ref{posError}] ($x$,$y$ Koordinaten) und $h$ den summierten Fehler der Orientierung [\ref{orienError}] mithilfe des \textit{Weighted-Least-Squares} Verfahren berechnen (\textit{Erfüllt 1.}). $n$ und $m$ Gewichten, wie stark die einzelnen Fehlerarten (Position und Orientierung) in den Gesamtfehler für die gegebenen Funktionsparameter $p$ beeinflussen.\\
Um die erhaltene Polynome einschränken zu können wurde $F$ noch gemäß der \textit{Tikhonov Regularisierung} \cite{kaipio2006statistical} angepasst. Durch die \textit{Tikhonov Regularisierung} können wenig gekrümmte Kurven bevorzugt werden, was für einen ruhigeren Fahrtverlauf sorgen kann.
\begin{ownequation}[H]
\begin{equation}
\label{minimizeFunction}
F(p) = F(p,x,y,s,w,n,m,tau) = n \cdot g(p,x,y,w) + m \cdot h(p,x,s,w) + tau \cdot p
\end{equation}
\begin{equation}
\label{posError}
g(p,x,y,w) = \sum_{i} w_i \cdot (f(p,x_i)-y_i)^2
\end{equation}
\begin{equation}
\label{orienError}
h(p,x,s,w) = \sum_{i} w_i \cdot (f'(p,x_i)-s_i)^2
\end{equation}
\caption{Zusammensetzung der Funktion F, die minimiert wird.}
\label{F-function}
\end{ownequation}
\todo{neute Grafiken auch mit Translation}
Um das Problem 3. zu lösen betrachten wir Abbildung \ref{prob3}. Hierbei sind die erkannten Punkte so angeordnet, dass keine Funktion $f(x)$ gefunden werden kann, um die Punkte sinnvoll zu verbinden. Zum einen laufen die Punkte teilweise parallel zur $Y-Achse$ (in diesem Fall müsste die Steigung unendlich groß sein) und zum anderen gibt es Werte $y_1 \neq y_2$, sodass gelten müsste $y_1 = f(x_n) = y_2$ für ein bestimmtes $n$. Dies verstößt jedoch gegen die Definition einer Polynomialfunktion.\\
\begin{figure}[H]
\includegraphics[scale=0.45]{curveFitting/pointsProblem.jpg}
\caption{Problem 3 vom Curve Fitting}
\label{prob3}
\end{figure}
In Abbildung \ref{prob3Marked} sind zwei Bereiche gekennzeichnet. Über den Bereich im schwarzen Kasten lässt sich noch gut eine Kurve legen (\ref{prob3Marked_a}). Kommt jedoch der Rote Bereich dazu würde eine Kurve über alle Punkte die Daten nicht mehr richtig abbilden.

\begin{figure}
\begin{tabular}{l}
\subfloat[Gesamter Datensatz in Bereiche unterteilt.]{\includegraphics[scale=0.45]{curveFitting/pointsProblemMarks.jpg}\label{prob3Marked_a}}\\
\subfloat[Kurve durch den blauen Bereich. Die Kurve lässt sich nicht im roten Bereich fortführen.]{\includegraphics[scale=0.45]{curveFitting/pointsProblemMarksWithCurve.jpg}\label{prob3Marked_b}}
\end{tabular}
\caption{Detektionsdaten in Bereiche unterteilt.}
\label{prob3Marked}
\end{figure}

Um dieses Problem zu lösen werden die Punkte in Teilbereiche unterteilt (siehe \ref{solve3}). Mit jedem neu von der Bilderkennung erhaltenen Punkt wird eine neue Kurve mit dem beschriebenen Verfahren berechnet. Die so entstandene Kurve wird dann mithilfe der neuesten Punkte geprüft. Sollte der Fehler in diesen Punkten einen Schwellwert überschreiten gilt der Bereich als abgeschlossen und ein neuer Bereich wird begonnen. Dieser neue Bereich zeichnet sich durch eine Rotation der $X-Achse$ um den Wert der Ableitung der letzten Funktion $f'(x_m)$ an der letzten Stelle $x_m$ aus.\\
Neben der Rotation wird der neue Bereich noch so verschoben, dass der transformierte Koordinatenursprung im letzten erkannten Punkt liegt. Diese Maßnahme ist nötig, um die Punkte stets in auf möglichst nah am Koordinatenursprung zu halten. Betrachtet man die Geraden in \todo{grafik}, die sich nur durch ihre Verschiebung auf der $X-Achse$ unterscheiden und versucht Polynome zweiten Grades durch diese Graden zu legen gibt es für jede Gerade zwei mögliche Formen der Polynome. Auch diese Polynome ähneln sich in ihrer Geometrie sehr, jedoch sind die Parameter sehr unterschiedlich. Vor allem der Schnittpunkt zur $Y-Achse$ liegt bei der verschobenen Gerade sehr weit auseinander. Daraus resultiert, dass je weiter der aktuell abzufahrende Kurvenabschnitt vom Koordinatenursprung entfernt liegt, der Parameterraum für gute Polynome wächst. Dadurch steigt die Wahrscheinlichkeit für die Minimierungsfunktion in ein lokales Minimum zu laufen und nicht mehr auf sich ändernde Verläufe reagieren zu können.

\begin{tikzpicture}[domain=0:20]
    \draw[very thin,color=gray] (-0.1,-1.1) grid (3.9,3.9);
    \draw[->] (-0.2,0) -- (4.2,0) node[right] {$x$};
    \draw[->] (0,-1.2) -- (0,4.2) node[above] {$f(x)$};
    \draw[color=red, domain=0:5] plot(\x,\x); 
    \draw[color=red, domain=3:8] plot(\x,\x+5); 
\end{tikzpicture}


In [\ref{solve3}] sind die Teilbereiche der Punkte und die Punkte in ihrem jeweiligen rotierten Koordinatensystem angegeben.\\

\begin{figure}
\begin{tabular}{l}
\subfloat[Rotierter Datensatz im neuen Koordinatensystem (schwarz). Grün gestrichelt ist das ursprüngliche Koordinatensystem angedeutet.]{\includegraphics[scale=0.45]{curveFitting/pointsProblemSolved.jpg}\label{solve3Marked_a}}\\
\subfloat[Zwei Kurven durch den kompletten Datensatz. Die grüne Kurve entspricht der Kurve aus \ref{prob3Marked_b}. Die blaue Kurveaus dem rotierten Koordinantensystem.]{\includegraphics[scale=0.45]{curveFitting/pointsProblemSolvedWithCurves.jpg}\label{solve3Marked_b}}
\end{tabular}
\caption{Lösungsansatz für Problem 3 zum Curve Fitting}
\label{prob3solvedMarked}
\end{figure}

Zum Rotieren der Punkte werden beim Erzeugen eines neuen Bereichs die Rotationsmatrix $actual\_rotation$, sowie die Inverse $actual\_rotation\_inv$ berechnet und Zwischengespeichert. Sollte es bereits eine Rotation geben wird die neue Rotation durch Multiplikation der alten Rotation mit der neuen erzeugt.\\
Sobald eine Transformation vorhanden ist werden alle Punkte vor der Regression in das rotierte Koordinatensystem transformiert. Durch dieses Verfahren kann stets eine Kurve entlang der $X-Achse$ berechnet werden.

\subsubsection*{Generierung der Gewichte}
\label{sec_calcWeights}
In der Gleichung [\ref{F-function}] werden die Gewichte zur Berechnung des Fehlers genutzt. Im Vektor \textit{w} befindet sich genau ein Gewicht für jeden Punkt. Dieses setzt sich zum einen aus den Merkmalen der Objekterkennung [\ref{sec_weights}] und zum Anderen aus der Höhe des AUVs und den Neigungswinkeln zusammen.\\
In der Gleichung [\ref{weightsEquation}] ist die Berechnung des Gewichts für einen Punkt abgebildet. Alle Faktoren für das Gewicht werden als Linearkombination mit den Vorfaktoren $g_1$ bis $g_8$ zusammengefasst. Die Vorfaktoren werden mithilfe von Testdaten gelernt. Hierfür werden Objekte mithilfe von Funktionen in die Simulationsumgebung gelegt. Das AUV wurde dann mithilfe der gleichen Funktionen am Objekt entlang gesteuert. Die erkannten Punkte der Objekterkennung, sowie die Pose des AUVs wurden dabei aufgezeichnet.\\
Für jeder dieser Testdaten wird dann die Funktion $F$ [\ref{F-function}] optimiert. Der Unterschied hierbei ist jedoch, dass aufgrund der Definition des Objektes anhand von Funktionen die Funktionsparameter $p$ bekannt sind. Somit wird $F$ nicht mehr über die Parameter, sondern über die Vorfaktoren der Gewichte $g$ optimiert. Die so gelernten Faktoren werden dann in den Livebetrieb übernommen.
\begin{ownequation}[H]
\begin{equation}
\begin{split}
w &= g_1 \cdot numParts \\
&+ g_2 \cdot area \\
&+ g_3 \cdot peakheight \\
&+ g_4 \cdot fitsBorder \\
&+ g_5 \cdot relativeCount \\
&+ g_6 \cdot auvheight \\
&+ g_7 \cdot roll \\
&+ g_8 \cdot yaw
\end{split}
\end{equation}
\caption{Zusammensetzung des Gewichts für einen Punkt.}
\label{weightsEquation}
\end{ownequation}