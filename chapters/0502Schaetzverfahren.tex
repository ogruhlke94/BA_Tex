\subsection{Testläufe}
Im folgenden werden die verschiedenen Testläufe genau beschrieben. Für alle Testläufe gilt, dass das AUV zuerst 30 Meter geradeaus fährt, bevor es auf das Objekt trifft, damit eine stabile Fahrt erreicht wird und die Schwankungen beim anfänglichen Beschleunigen die Ergebnisse nicht verfälschen. Ebenso wird auch gewährleistet, dass das AUV stets direkt auf das Objekt trifft, da dass Explorieren und Auffinden des Objektes nicht Teil der Arbeit ist.\\
Zu jedem Testlauf befindet sich auf der CD ein Video, in dem das AUV von oben, die Rohbilder der Kamera, die Ausgabe der Objekterkennung und das berechnetet Polynom zu sehen ist.
\subsubsection{Gerader Verlauf}
Für die ersten Tests wurde ein 100 Meter langes Objekt gerade in die Simulationsumgebung eingefügt. Dieses Objekt ist in mehreren Bereichen vom Meeresboden leicht bis komplett bedeckt.\\
Die Testläufe mit geraden Objekte zeigten, dass mit steigender Anzahl an detektierten Punkt die gerade immer bessert vom Schätzverfahren abgebildet wurde. Es sind bei einer dann auch längere Strecken ohne Sichtkontakt kein Problem.

\begin{figure}[H]
\begin{tabular}{cc}
\multicolumn{2}{c}{\subfloat[Fahrtverlauf des AUVs (rot) an einem geraden Objekt (blau). Nach erstem Sichtkontakt zum Objekt ist ein einpendeln auf die gerade Linie zu beobachten.]{\includegraphics[height=0.4\textheight,width=\textwidth]{/testlaeufe/gradeGut/auvroute.jpg}}}\\
\subfloat[Fehler der AUV Position zur echten Position des Objektes. Auch hier ist zu beobachten, dass ein großer Fehler zu Beginn des Objektes auftritt, der beim Fahrtverlauf weiter verringert wird.]{\includegraphics[height=0.3\textheight,width=0.5\textwidth]{/testlaeufe/gradeGut/groundTruthPosition.jpg}}&
\subfloat[Fehler der detektierten Objektposition zur echten Objektposition. In Betrachtung von \textit{b)} ist zu beobachten, dass der Fehler der detektierten Objektposition größer ist, als der Fehler im daraus resultierenden Fahrtverlauf.]{\includegraphics[height=0.3\textheight,width=0.5\textwidth]{/testlaeufe/gradeGut/groundTruth.jpg}}
\end{tabular}
\caption{Testlauf an einem geraden Objekt. Nach anfänglich größeren Fehler folgt das AUV dem Objekt mit nur sehr geringem Fehler. Das Einpendeln ist auf die Berechnug des Polynoms zurückzuführen, da bei wenigen Punkten zu Beginn der Verlauf noch nicht eindeutig als Gerade bestimmbar ist. Siehe hierfür Kapitel \ref{sec_pendel}}
\label{testStraight}
\end{figure}

\subsubsection{Kurve}
Nach den Tests zum geraden Verlauf wurden kurvige Objekte mithilfe von Polynomial- und Exponentialfunktionen in die Simulation eingefügt. Hierbei wurde darauf geachtet kein Polynom zweiten Grades zu verwenden, um dem Regressionsverfahren keine perfekte Lösung zu bieten.\\
Ziel dieser Tests ist es zu zeigen, dass das Folgen einer Links- sowie Rechtskurve und auch ein wechseln zwischen beiden Kurvenarten möglich sind. Für letzteres wurde eine Sinuskurve genutzt, um ein entsprechendes Objekt zu erzeugen.

Außerdem wurde noch eine Kurve nach einer langen Geraden erzeugt. Hiermit wird getestet, ob auch wechselnden geometrischen Strukturen gefolgt werden kann.\\

Bei den Tests mit kurvigen Objekten ist zu beobachten, dass zu Beginn jeder Kurve zunächst ein größerer Fehler zur Objektlage entsteht. Bei lang gezogenen Kurven wird dieser Fehler schnell wieder ausgeglichen. Auch bei mehreren Kurven ist dieses Verhalten zu beobachten (siehe \ref{testSCurve}).


\begin{figure}[H]
\begin{tabular}{cc}
\multicolumn{2}{c}{\subfloat[Fahrtverlauf (rot) bei einer Kurve (blau).]{\includegraphics[height=0.4\textheight,width=\textwidth]{/testlaeufe/linkskurve/auvroute.jpg}}}\\
\subfloat[Fehler der AUV Position zur echten Position des Objektes. Am Ende ist zu beobachten, wie sich der systematische Fehler aus \textit{c)} in einem beständigen Fehler der Fahrt resultiert.]{\includegraphics[height=0.3\textheight,width=0.5\textwidth]{/testlaeufe/linkskurve/groundTruthPosition.jpg}}&
\subfloat[Fehler der detektierten Objektposition zur echten Objektposition. Es scheint, dass die zweite Hälfte der Punkte einen systematischen Fehler hat. Siehe hierfür Kapitel \ref{sec_sysError}.]{\includegraphics[height=0.3\textheight,width=0.5\textwidth]{/testlaeufe/linkskurve/groundTruth.jpg}}
\end{tabular}
\caption{Testlauf mit einer Kurve. In \textit{a)} und \textit{b)} ist zu erkennen, dass einige Meter benötigt werden, um auf die Kurve zu reagieren. Der zweite größere Fehlerausschlag ist durch eine teilweise komplette Verdeckung des Objektes zu erklären. In \textit{a)} ist sehr gut zu beobachten, dass der Fehler zuerst ansteigt, sobald das Objekt nicht sichtbar ist, bei erneuter Detektion des Objektes aber sehr schnell korrigiert wird.}
\label{fig_leftCurve}
\end{figure}

\begin{figure}[H]
\begin{tabular}{cc}
\multicolumn{2}{c}{\subfloat[Fahrtverlauf (rot) bei einer Kurve (blau).]{\includegraphics[height=0.4\textheight,width=\textwidth]{/testlaeufe/rechtskurveLinks/auvroute.jpg}}}\\
\subfloat[Fehler der AUV Position zur echten Position des Objektes.]{\includegraphics[height=0.3\textheight,width=0.5\textwidth]{/testlaeufe/rechtskurveLinks/groundTruthPosition.jpg}}&
\subfloat[Fehler der detektierten Objektposition zur echten Objektposition.]{\includegraphics[height=0.3\textheight,width=0.5\textwidth]{/testlaeufe/rechtskurveLinks/groundTruth.jpg}}
\end{tabular}
\caption{Testlauf mit einer Kurve. In diesem Lauf wurde die Kurve in die andere Richtung wie in Abb. \ref{fig_leftCurve} erzeugt. Wie zu erwarten sind die Ergebnisse in diesem Lauf analog zur anderen Kurve.}
\label{fig_rightCurve}
\end{figure}

\begin{figure}[H]
\begin{tabular}{cc}
\multicolumn{2}{c}{\subfloat[Fahrtverlauf (rot) bei einer Kurve nach einer längeren Gerade (blau).]{\includegraphics[height=0.4\textheight,width=\textwidth]{/testlaeufe/gradeKurveSicht/auvroute.jpg}}}\\
\subfloat[Fehler der AUV Position zur echten Position des Objektes.]{\includegraphics[height=0.3\textheight,width=0.5\textwidth]{/testlaeufe/gradeKurveSicht/groundTruthPosition.jpg}}&
\subfloat[Fehler der detektierten Objektposition zur echten Objektposition.]{\includegraphics[height=0.3\textheight,width=0.5\textwidth]{/testlaeufe/gradeKurveSicht/groundTruth.jpg}}
\end{tabular}
\caption{Testlauf mit einer Kurve nach einer längeren Gerade. In diesem Lauf wird der Wechsel zwischen Gerade und Kurve getestet. Hierbei ist zu sehen, dass die gerade Strecke zunächst gut verfolgt wird. Der Wechsel zur Kurve resultiert in einem größeren Fehler und einem einpendeln. Dieses Einpendeln fällt stärker aus, da die Parameter so gewählt werden mussten, dass schnell auf Kurven reagiert werden kann (vgl. \ref{sec_param})}
\label{fig_rightCurve}
\end{figure}

\begin{figure}[H]
\begin{tabular}{cc}
\multicolumn{2}{c}{\subfloat[Fahrtverlauf (rot) bei einer Sinuskurve (blau).]{\includegraphics[height=0.4\textheight,width=\textwidth]{/testlaeufe/sinusSicht/auvroute.jpg}}}\\
\subfloat[Fehler der AUV Position zur echten Position des Objektes. Eine interessante Beobachtung in dieser Grafik ist der sehr ähnliche Fehlerausschlag ]{\includegraphics[height=0.3\textheight,width=0.5\textwidth]{/testlaeufe/sinusSicht/groundTruthPosition.jpg}}&
\subfloat[Fehler der detektierten Objektposition zur echten Objektposition.]{\includegraphics[height=0.3\textheight,width=0.5\textwidth]{/testlaeufe/sinusSicht/groundTruth.jpg}}
\end{tabular}
\caption{Beim Testlauf mit der Sinuskurve ist zu beobachten, dass innerhalb der Kurven aufgrund der Richtungsänderung des Verlaufs einen größeren Fehler der Verfolgung gibt. Nach der Kurve wird dem Objekt jedoch schnell wieder gut gefolgt.}
\end{figure}

\begin{figure}[H]
\begin{tabular}{cc}
\multicolumn{2}{c}{\subfloat[Fahrtverlauf (rot) bei einem kurvigen Objektverlauf(blau). Da die Kurve zu Beginn einen starken Knick macht ist dort ein größerer Fehler, bis richtig reagiert wird.]{\includegraphics[height=0.4\textheight,width=\textwidth]{/testlaeufe/S-Kurve_Gut/auvroute.jpg}}}\\
\subfloat[Fehler der AUV Position zur echten Position des Objektes. Trotz des Fehlers im geraden Bereich und dem sehr großen Fehler innerhalb der Rechtskurve wird das Objekt nach dem Ausschlag wieder gut verfolgt.]{\includegraphics[height=0.3\textheight,width=0.5\textwidth]{/testlaeufe/S-Kurve_Gut/groundTruthPosition.jpg}}&
\subfloat[Fehler der detektierten Objektposition zur echten Objektposition. Der hier zu beobachtete Fehler ist im gesamten Bereich hoch.]{\includegraphics[height=0.3\textheight,width=0.5\textwidth]{/testlaeufe/S-Kurve_Gut/groundTruth.jpg}}
\end{tabular}
\caption{Die Kurven in diesem Lauf sind für das AUV aufgrund der starken Krümmung schwer zu folgen. Im mittleren Bereich ist auf längerer Strecke ein größerer Fehler. Aufgrund dieses Fehlers wird die Rechtskurve fast \textit{verpasst} jedoch aufgrund des Schätzverfahrens trotzdem noch verfolgt. In diesem Lauf ist der Zusammenhang zwischen Positions- und Detektionsfehler deutlich zu erkennen.}
\label{testSCurve}
\end{figure}

\subsubsection{Kreisbahn}
Für die finalen Tests wurden Kreisbahnen in Form von Ellipsen verwendet. Eine Ellipse erfüllt einige Eigenschaften, die für die Arbeit nicht trivial zu lösen sind. Zum einen gibt es verschieden stark gebogene Kurven und fast gerade Abschnitte. Zum anderen gibt es ständig Abschnitte, die parallel zur $Y-Achse$ verlaufen.\\
In diesen Tests ist zu sehen, dass sowohl Kreise, als auch der Wechsel zwischen Geraden und Kreisabschnitten gefolgt werden kann.

\begin{figure}[H]
\begin{tabular}{cc}
\multicolumn{2}{c}{\subfloat[Fahrtverlauf (rot) bei einem Kreis (blau). Es wurden anderthalb runden im Kreis gefahren.]{\includegraphics[height=0.4\textheight,width=\textwidth]{/testlaeufe/kreissicht/auvroute.jpg}}}\\
\subfloat[Fehler der AUV Position zur echten Position des Objektes. Es ist ein gleichmäßiges Auftreten von Fehlerspitzen zu beobachten. Der größte Ausschlag ist einer Unsichtbarkeit des Objektes innerhalb des rechten oberen Kreisabschnitts zuzuschreiben.]{\includegraphics[height=0.3\textheight,width=0.5\textwidth]{/testlaeufe/kreissicht/groundTruthPosition.jpg}}&
\subfloat[Fehler der detektierten Objektposition zur echten Objektposition. Es sind zwei Bereiche mit größerem Fehler zu beobachten. Diese liegen beide im unteren linken Bereich des Kreises, in dem das Objekt teilweise vom Meeresboden bedeckt ist.]{\includegraphics[height=0.3\textheight,width=0.5\textwidth]{/testlaeufe/kreissicht/groundTruth.jpg}}
\end{tabular}
\caption{Im Testlauf der Kreisbahn ist zu beobachten, wie die ständig ändernde Krümmung der Bahn zu Fehlerspitzen führt. Bei jeder Spitze ist der Fehler der Regression so hoch, dass eine Transformation der detektierten Punkte stattfindet (siehe Kapitel \ref{sec_curveFit}), die zu einer direkten Abnahme des Fehlers führt.}
\end{figure}

\begin{figure}[H]
\begin{tabular}{cc}
\multicolumn{2}{c}{\subfloat[Fahrtverlauf (rot) bei einer Kreisbahn mit geraden Abschnitten (blau).]{\includegraphics[height=0.4\textheight,width=\textwidth]{/testlaeufe/gradeKreissicht/auvroute.jpg}}}\\
\subfloat[Fehler der AUV Position zur echten Position des Objektes.]{\includegraphics[height=0.3\textheight,width=0.5\textwidth]{/testlaeufe/gradeKreissicht/groundTruthPosition.jpg}}&
\subfloat[Fehler der detektierten Objektposition zur echten Objektposition.]{\includegraphics[height=0.3\textheight,width=0.5\textwidth]{/testlaeufe/gradeKreissicht/groundTruth.jpg}}
\end{tabular}
\caption{Testlauf von einer Kreisbahn gemischt mit geraden Abschnitten. Beim Übergang in die Kreisbahn ist ein erwarteter hoher Fehler zu beobachten, der durch den Wechsel der Form zu erklären ist. Innerhalb der Kreisbahn sind einige verdeckte Bereiche, was die Fehler in \textit{b)} und \textit{c)} erklärt.}
\label{testStraightCirc}
\end{figure}

\subsubsection{Schlechte Sichtbedingungen}
Einige Tests wurden mit sehr schlechten Sichtbedingungen (siehe Abb. \ref{img_badSeight}) wiederholt. In den Tests ist zu sehen, dass selbst unter diesen Bedingungen dem Objekt gefolgt werden kann. Jedoch ist bei diesen Tests zu beachten, dass der Templateschwellwert sehr gering gewählt wurde und dadurch eine sehr sensitive Objekterkennung genutzt wurde. Da jedoch in der Simulation nur Meeresboden und Zielobjekt existieren führte dies nicht zu Problemen. Schon kleinste Objekte, wie kleinere Felsen oder Pflanzen, würden bei einem solch geringen Schwellwert zu schwerwiegenden Fehldetektionen führen. Tests mit anderen Objekten in der Simulationsumgebung wurden jedoch nicht durchgeführt, da es keine einfache Methode gibt Steine oder Pflanzen hinzuzufügen. Die Tatsache, dass aber selbst der Meeresboden schon zu Störpunkten innerhalb des Binärbilds führt unterstützt diese Aussage.

\begin{figure}[H]
\begin{tabular}{cc}
\multicolumn{2}{c}{\subfloat[Fahrtverlauf des AUVs (rot) bei einer Kurve (blau) unter schlechten Sichtbedingungen. ]{\includegraphics[height=0.4\textheight,width=\textwidth]{/testlaeufe/linkskurveschlechtesicht/auvroute.jpg}}}\\
\subfloat[Fehler der AUV Position zur echten Position des Objektes.]{\includegraphics[height=0.3\textheight,width=0.5\textwidth]{/testlaeufe/linkskurveschlechtesicht/groundTruthPosition.jpg}}&
\subfloat[Fehler der detektierten Objektposition zur echten Objektposition. Der Anstieg zum Ende ist auf leichte Verdeckung des Objektes zurückzuführen.]{\includegraphics[height=0.3\textheight,width=0.5\textwidth]{/testlaeufe/linkskurveschlechtesicht/groundTruth.jpg}}
\end{tabular}
\caption{Das Folgen der Kurve funktioniert auch bei schlechten Sichtbedingungen. Es ist jedoch zu beobachten, dass eine nur leichte Verdeckung des Objektes einen starken Einfluss auf den Fehler hat.}
\label{curveBadSight}
\end{figure}

\begin{figure}[H]
\begin{tabular}{cc}
\multicolumn{2}{c}{\subfloat[Fahrtverlauf (rot) bei einer Kreisbahn (blau) unter schlechten Sichtbedingungen.]{\includegraphics[height=0.4\textheight,width=\textwidth]{/testlaeufe/kreisschlechtesicht/auvroute.jpg}}}\\
\subfloat[Fehler der AUV Position zur echten Position des Objektes.]{\includegraphics[height=0.3\textheight,width=0.5\textwidth]{/testlaeufe/kreisschlechtesicht/groundTruthPosition.jpg}}&
\subfloat[Fehler der detektierten Objektposition zur echten Objektposition.]{\includegraphics[height=0.3\textheight,width=0.5\textwidth]{/testlaeufe/kreisschlechtesicht/groundTruth.jpg}}
\end{tabular}
\caption{Ähnlich zu Abb. \ref{curveBadSight} wird dem Objekt gut gefolgt und leichte Verdeckungen haben einen starken Einfluss auf den Fehler.}
\end{figure}