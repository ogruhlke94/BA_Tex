\subsection{Parametrisierung}
Während der Tests ist stark aufgefallen, dass das Regressionsverfahren sehr parameterabhängig ist. Wie im vorherigen Kapitel zu sehen ist funktioniert die implementierte Lösung mit der richtigen Parametrisierung ein gutes Verhalten liefert. 
Bei falscher Parametrisierung wird die Verfolgung sehr viel schlechter bis unmöglich.\\
Große Unterschiede gibt es bei geraden und kurvigen Objekten. Als ausschlaggebende Parameter stellten sich die Gewichtungen der einzelnen Fehlerarten, die Stärke \textit{Tikhonov Regularisierung} (siehe Gleichung \ref{F-function} und der maximale Gesamtfehler der Regression bis zu einer Transformation (siehe Listing \ref{transPseudo}) heraus.
Bei geraden Objekten führt eine Gleichgewichtung der Fehlerarten, eine Starke \textit{Tikhonov Regularisierung} und ein hoher erlaubter Maximalfehler zu sehr guten Ergebnissen (vgl. \ref{sec_pendel}).\\
Kurvige Objekte, wie in Abb. \ref{testSCurve} oder \ref{testStraightCirc} werden am besten bei einer höheren Gewichtung des Orientierungsfehlers, keine \textit{Tikhonov Regularisierung} und ein geringerer erlaubter Maximalfehler zu guten Ergebnissen.
Problematisch ist, dass die \textit{falsche} Parametrisierung (z.B. hoher Maximalfehler für kurvige Objekte) im schlimmsten Fall zu einem großen Fehler, damit einhergehender Sichtverlust zum Objekt und einem schlecht geschätzten Polynom, dass keine erneute Annäherung zum Objekt mehr gelingt.\todo{hier fehlgeschlagene tests}

\subsubsection{Einpendeln}
\label{sec_pendel}
In jedem Testlauf fiel auf, dass bei erster Sicht des Objektes eine Art \texttt{Einpendeln} stattfindet, also ein zuerst großer Fehler, der dann über mehrere Meter beständig abnimmt, bis eine stabile Fahrt über dem Objekt erreicht wird. Diese Beobachtung konnte auch beim Wechsel von geraden Objektverläufen auf kurvige Verläufe gemacht werden.\\
In Abbildung \ref{figpendel} ist diese Beobachtung mit verschiedenen Parametrisierungen dargestellt.

\begin{figure}[H]
\begin{tabular}{ccc}
\subfloat[Sehr gute Parametrisierung mit Gleichgewichtung der Fehlerarten, \textit{Tikhonov Regularisierung} und hohem Maximalfehler.]{\includegraphics[width=0.3\textwidth,height=0.3\textheight]{/testlaeufe/gradeGut/groundTruthPosition.jpg}}&
\subfloat[Gute Parametrisierung mit Gleichgewichtung der Fehlerarten, geringer \textit{Tikhonov Regularisierung} und weder hohem, noch geringen Maximalfehler.]{\includegraphics[width=0.3\textwidth,height=0.3\textheight]{/testlaeufe/Gradeok/groundTruthPosition.jpg}}&
\subfloat[Schlechte Parametrisierung mit höherer Gewichtung des Orientierungsfehlers, keiner \textit{Tikhonov Regularisierung} und geringen Maximalfehler.]{\includegraphics[width=0.3\textwidth,height=0.3\textheight]{/testlaeufe/Gradeschlecht/groundTruthPosition.jpg}}
\end{tabular}
\caption{Testlauf mit einem geraden Objekt [Abb. \ref{testStraight}] mit unterschiedlicher Parametrisierung. Im Fehler der AUV-Position zur Objektposition ist zu erkennen, dass bei den guten Parametrisierungen der Fehler über die Zeit abfällt. Bei der sehr guten Parametrisierung (\textit{a)}) sogar noch weitaus schneller. Bei der schlechten Parametrisierung (\textit{c)}) findet keine Verringerung des Fehlers statt. Die Höhe der Ausschläge nehmen auch nicht beständig ab, wie bei den guten Parametern.}
\label{figpendel}
\end{figure}

\subsection{Systematischer Fehler}
\label{sec_sysError}